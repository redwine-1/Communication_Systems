\section*{Theory}
\addcontentsline{toc}{section}{Theory}

Amplitude modulation is a process where the amplitude of a carrier
wave is varied about a mean value linearly with the message signal. Amplitude modulated signal:

\begin{equation}
    s(t) = A_c[1 + k_a m(t)]\cos(2\pi f_c t)
\end{equation}

where,\\
\begin{itemize}
    \item $s(t)$ is the modulated signal.
    \item $A_c$ is the carrier amplitude.
    \item $m(t)$ is the message signal.
    \item $f_c$ is the carrier frequency.
    \item $k_a$ is the amplitude sensitivity of the modulator.
\end{itemize}

For single-tone modulation, the message signal is given by:
\begin{equation}
    m(t) = A_m \cos(2\pi f_m t)
\end{equation}

Thus, the modulated signal becomes:
\begin{equation}
    s(t) = A_c[1 + \mu \cos(2\pi f_m t)]\cos(2\pi f_c t) \\
\end{equation}
\begin{equation}
    s(t) = A_c\cos(2\pi f_c t) + \frac{A_c\mu}{2}\cos\{2\pi (f_c + f_m) t\} + \frac{A_c\mu}{2}\cos\{2\pi (f_c - f_m) t\}
    \label{eq:single_am}
\end{equation}

where,\\
$\mu = k_a A_m$ is the modulation index.
For Conventional AM, the modulation index must be less than 1.\\

The fourier transform of the modulated signal is given by:
\[
    s(f)=\frac{A_c}{2}[\delta(f+f_c) + \delta(f-f_c)] + \frac{\mu A_c}{4}[\delta(f+f_c+f_m) + \delta(f-f_c-f_m)] 
\]
\begin{equation}
    \frac{\mu A_c}{4}[\delta(f+f_c-f_m) + \delta(f-f_c+f_m)]
    \label{eq:single_am_fft}
\end{equation}

